\documentclass[]{article}
\usepackage{lmodern}
\usepackage{amssymb,amsmath}
\usepackage{ifxetex,ifluatex}
\usepackage{fixltx2e} % provides \textsubscript
\ifnum 0\ifxetex 1\fi\ifluatex 1\fi=0 % if pdftex
  \usepackage[T1]{fontenc}
  \usepackage[utf8]{inputenc}
\else % if luatex or xelatex
  \ifxetex
    \usepackage{mathspec}
  \else
    \usepackage{fontspec}
  \fi
  \defaultfontfeatures{Ligatures=TeX,Scale=MatchLowercase}
\fi
% use upquote if available, for straight quotes in verbatim environments
\IfFileExists{upquote.sty}{\usepackage{upquote}}{}
% use microtype if available
\IfFileExists{microtype.sty}{%
\usepackage{microtype}
\UseMicrotypeSet[protrusion]{basicmath} % disable protrusion for tt fonts
}{}
\usepackage[margin=1in]{geometry}
\usepackage{hyperref}
\hypersetup{unicode=true,
            pdftitle={Basic Statistics in R},
            pdfauthor={Sandra Erdmann},
            pdfborder={0 0 0},
            breaklinks=true}
\urlstyle{same}  % don't use monospace font for urls
\usepackage{color}
\usepackage{fancyvrb}
\newcommand{\VerbBar}{|}
\newcommand{\VERB}{\Verb[commandchars=\\\{\}]}
\DefineVerbatimEnvironment{Highlighting}{Verbatim}{commandchars=\\\{\}}
% Add ',fontsize=\small' for more characters per line
\usepackage{framed}
\definecolor{shadecolor}{RGB}{248,248,248}
\newenvironment{Shaded}{\begin{snugshade}}{\end{snugshade}}
\newcommand{\KeywordTok}[1]{\textcolor[rgb]{0.13,0.29,0.53}{\textbf{#1}}}
\newcommand{\DataTypeTok}[1]{\textcolor[rgb]{0.13,0.29,0.53}{#1}}
\newcommand{\DecValTok}[1]{\textcolor[rgb]{0.00,0.00,0.81}{#1}}
\newcommand{\BaseNTok}[1]{\textcolor[rgb]{0.00,0.00,0.81}{#1}}
\newcommand{\FloatTok}[1]{\textcolor[rgb]{0.00,0.00,0.81}{#1}}
\newcommand{\ConstantTok}[1]{\textcolor[rgb]{0.00,0.00,0.00}{#1}}
\newcommand{\CharTok}[1]{\textcolor[rgb]{0.31,0.60,0.02}{#1}}
\newcommand{\SpecialCharTok}[1]{\textcolor[rgb]{0.00,0.00,0.00}{#1}}
\newcommand{\StringTok}[1]{\textcolor[rgb]{0.31,0.60,0.02}{#1}}
\newcommand{\VerbatimStringTok}[1]{\textcolor[rgb]{0.31,0.60,0.02}{#1}}
\newcommand{\SpecialStringTok}[1]{\textcolor[rgb]{0.31,0.60,0.02}{#1}}
\newcommand{\ImportTok}[1]{#1}
\newcommand{\CommentTok}[1]{\textcolor[rgb]{0.56,0.35,0.01}{\textit{#1}}}
\newcommand{\DocumentationTok}[1]{\textcolor[rgb]{0.56,0.35,0.01}{\textbf{\textit{#1}}}}
\newcommand{\AnnotationTok}[1]{\textcolor[rgb]{0.56,0.35,0.01}{\textbf{\textit{#1}}}}
\newcommand{\CommentVarTok}[1]{\textcolor[rgb]{0.56,0.35,0.01}{\textbf{\textit{#1}}}}
\newcommand{\OtherTok}[1]{\textcolor[rgb]{0.56,0.35,0.01}{#1}}
\newcommand{\FunctionTok}[1]{\textcolor[rgb]{0.00,0.00,0.00}{#1}}
\newcommand{\VariableTok}[1]{\textcolor[rgb]{0.00,0.00,0.00}{#1}}
\newcommand{\ControlFlowTok}[1]{\textcolor[rgb]{0.13,0.29,0.53}{\textbf{#1}}}
\newcommand{\OperatorTok}[1]{\textcolor[rgb]{0.81,0.36,0.00}{\textbf{#1}}}
\newcommand{\BuiltInTok}[1]{#1}
\newcommand{\ExtensionTok}[1]{#1}
\newcommand{\PreprocessorTok}[1]{\textcolor[rgb]{0.56,0.35,0.01}{\textit{#1}}}
\newcommand{\AttributeTok}[1]{\textcolor[rgb]{0.77,0.63,0.00}{#1}}
\newcommand{\RegionMarkerTok}[1]{#1}
\newcommand{\InformationTok}[1]{\textcolor[rgb]{0.56,0.35,0.01}{\textbf{\textit{#1}}}}
\newcommand{\WarningTok}[1]{\textcolor[rgb]{0.56,0.35,0.01}{\textbf{\textit{#1}}}}
\newcommand{\AlertTok}[1]{\textcolor[rgb]{0.94,0.16,0.16}{#1}}
\newcommand{\ErrorTok}[1]{\textcolor[rgb]{0.64,0.00,0.00}{\textbf{#1}}}
\newcommand{\NormalTok}[1]{#1}
\usepackage{graphicx,grffile}
\makeatletter
\def\maxwidth{\ifdim\Gin@nat@width>\linewidth\linewidth\else\Gin@nat@width\fi}
\def\maxheight{\ifdim\Gin@nat@height>\textheight\textheight\else\Gin@nat@height\fi}
\makeatother
% Scale images if necessary, so that they will not overflow the page
% margins by default, and it is still possible to overwrite the defaults
% using explicit options in \includegraphics[width, height, ...]{}
\setkeys{Gin}{width=\maxwidth,height=\maxheight,keepaspectratio}
\IfFileExists{parskip.sty}{%
\usepackage{parskip}
}{% else
\setlength{\parindent}{0pt}
\setlength{\parskip}{6pt plus 2pt minus 1pt}
}
\setlength{\emergencystretch}{3em}  % prevent overfull lines
\providecommand{\tightlist}{%
  \setlength{\itemsep}{0pt}\setlength{\parskip}{0pt}}
\setcounter{secnumdepth}{0}
% Redefines (sub)paragraphs to behave more like sections
\ifx\paragraph\undefined\else
\let\oldparagraph\paragraph
\renewcommand{\paragraph}[1]{\oldparagraph{#1}\mbox{}}
\fi
\ifx\subparagraph\undefined\else
\let\oldsubparagraph\subparagraph
\renewcommand{\subparagraph}[1]{\oldsubparagraph{#1}\mbox{}}
\fi

%%% Use protect on footnotes to avoid problems with footnotes in titles
\let\rmarkdownfootnote\footnote%
\def\footnote{\protect\rmarkdownfootnote}

%%% Change title format to be more compact
\usepackage{titling}

% Create subtitle command for use in maketitle
\newcommand{\subtitle}[1]{
  \posttitle{
    \begin{center}\large#1\end{center}
    }
}

\setlength{\droptitle}{-2em}

  \title{Basic Statistics in R}
    \pretitle{\vspace{\droptitle}\centering\huge}
  \posttitle{\par}
    \author{Sandra Erdmann}
    \preauthor{\centering\large\emph}
  \postauthor{\par}
    \date{}
    \predate{}\postdate{}
  

\begin{document}
\maketitle

\section{Section}\label{section}

\subsection{Basics}\label{basics}

To create a \textbf{Markdown} file, click on

\subsubsection{Short cuts}\label{short-cuts}

Control+Alt+i : insert new chunk

Control+Shift+Enter : run current chunk

Control+Shift+m : \%\textgreater{}\%

\subsubsection{Equations}\label{equations}

\begin{Shaded}
\begin{Highlighting}[]
\CommentTok{# $$ mu = 5 $$}
\end{Highlighting}
\end{Shaded}

\[ mu = 5 \]

\subsubsection{Load datasets}\label{load-datasets}

To load a dataset, you type : dataset\textless{}-
read.csv(`C:/Users/admin/Desktop/R-Course by
Murray/07092018/dataset.csv', strip.white=T)

strip.white=T means, that you exclude all spaces in your excel file to
avoid complications.

For import, simply run the chunk, which contains your dataset

\subsubsection{Github}\label{github}

Is a platform to store your latest R Studio versions and if you wish to
work on R projects with others.

Follow these instructions: \url{http://r-pkgs.had.co.nz/git.html}

Install Git first with this link: \url{http://git-scm.com/download/win}.

Click on Tools and choose global options. Click on Git/SVN and click on
`Create RSA key'. Add the key to \url{https://github.com/settings/ssh}.

Now create a new repository. Click on Tools and then project options,
Click on Git/SVN and chose version control GIT

You can create new repositories, create new branches, make and commit
branches,

\subsection{Statistical terms}\label{statistical-terms}

\url{https://select-statistics.co.uk/resources/glossary-page/}

\textbf{degrees of freedom} :

\textbf{Chi-Square} :

\textbf{confidence intervals} :

\textbf{Maximum Likelihood} : gets used in generalized linear models

\textbf{Multicolinearity} : When there are linear relationships
(correlations) between two or more of the potential drivers; this can
lead to difficulty in the interpretation of the model coefficients.
Example: a questionnaire might ask whether the staff were friendly, and
also whether they were helpful, which we would expect to be highly
related. There are various statistical approaches that can be used to
deal with multicollienarity, including the use of principal component
analysis PCA to reduce the number of potential drivers to a set of
linearly uncorrelated variables.

\textbf{OLS} : Least Squares; gets used in linear models

\textbf{PCA} : principal component analysis

\textbf{R Square} :

\subsection{Libraries}\label{libraries}

To install libraries in R type: install.packages(``name\_of\_library'')

To load libraries in R Studio type: library(name\_of\_library)

\textbf{library(car)} : for regressions

\textbf{library(emmeans)} : for the function emmeans

\textbf{library(lme4)} : to fit lme and lmer

\textbf{library(MuMIn)} : to compare AIC's

\textbf{library(rstanarm)} : for Bayesian models

\textbf{library(tidyverse)} : contains main functions, such as NOTE:
Always load the the end to avoid masking of standard functions

\textbf{library(vegan)} : for PCA's

\subsection{Functions}\label{functions}

\textbf{abline(data.model)} : to fit a regression line through your
model

\subsubsection{Data wrangling}\label{data-wrangling}

\textbf{mutate} :

\textbf{head(data)} : displays the first 6 rows of your dataset

\subsubsection{something else}\label{something-else}

\subsection{Plots}\label{plots}

\textbf{plot}

\textbf{ggplot}

\textbf{homogeneity} :

The \textbf{box plot} (a.k.a. box and whisker diagram) is a standardized
way of displaying the distribution of data based on the five number
summary: minimum, first quartile, median, third quartile, and maximum

\textbf{scatterplot}

\subsection{Which model to choose?}\label{which-model-to-choose}

\url{https://help.xlstat.com/customer/en/portal/articles/2062461-which-statistical-model-should-you-choose-?b_id=9283}

\url{http://www.dataanalytics.org.uk/Data\%20Analysis/Statistics/choosing-your-stats-test.htm}

Fitting a model to data means choosing the statistical model that
predicts values as close as possible to the ones observed in your
population.

There are different methods that you can use to do so. The best approach
is to cross-validate and choose the model with the highest R² , which
measures how much of the total variability in your data is accounted for
by the model. However, as it is global measure of fit, a high R² doesn't
guarantee a valid model. Therefore, the main tool used is the Residual
Analysis, which gives a more immediate and clear illustration of the
relationship between the model and the data used. While doing a
statistial analysis, it's very important to make sure about the Godness
of Fit of the model used.

\begin{enumerate}
\def\labelenumi{(\arabic{enumi})}
\item
  Fit the model
\item
  Validate the model
\end{enumerate}

validation is looking at the residuals vs.~standardized residuals,
Cook's distance, etc. If you deny the model, because these values don't
look good, check for the lack of fit, which is a comparison to a chi
square. The value that should be greater than 0.5.

\subsection{Gaussian Model}\label{gaussian-model}

Assumptions: - normal distributed data

\subsection{Poisson Model}\label{poisson-model}

Assumptions:

If the plot shows that there is a relationship between the mean and the
variance, which you can see by the high noise near the low values and
the rare dots at high values (called wedge), you cannot use models
assuming a normal distribution, because it is a massive violation of
linear regression.

Often non-normality and unequal variance are linked: if we fix one, we
can fix the other. So we can log transform our data to make the dataset
having a normal distribution. Log transformation is possible if there
are no zeros, we can check by the function summary and check for the
lowest value

\subsection{Binomial model}\label{binomial-model}

\subsection{Negative Binomial}\label{negative-binomial}

\subsection{Beta}\label{beta}

\subsection{Beta zero inflated}\label{beta-zero-inflated}

\subsection{Gamma}\label{gamma}

\subsection{ANOVA}\label{anova}

\subsection{ANCOVA}\label{ancova}

Assumptions:

\begin{itemize}
\tightlist
\item
  linearity of regression
\item
  homogeneity of error variances
\item
  independence of error terms
\item
  normality of error terms
\item
  homogeneity of error slopes
\end{itemize}

\subsection{Multivariate Analysis or
PCA}\label{multivariate-analysis-or-pca}

Assumptions:

linearity and normality (not for Q-Mode, only for R-Mode)

community questions useful for correllated things, because the idea is
to combine things in order to get patterns if there are the same
abundances of plant, they might have the same response to some aspect,
so we are interested in the underliying drivers and conditions that
cause those correllations

Terminology \textbf{R Mode}: combine data on the basis of correlation,
PCA, Principle coordinate analysis or principal correspondant analysis;
the resulted variable from the combined variable can be used in any
other analysis (univariate analysis), hypothesis test

disadvantage/ assumptions: linearity and normality, likke running
regressions

Alternative/ \textbf{Q Mode}: is multidimensional scaling, anasim,
permanova, work on the rows and not the columns; implication is the
comnbinations you get they are not independent number, so cnnot be used
in analysis like regression or mixed effects model or anything similar
to that, because they are not independent to another on the plus side,
flexible in the patterns things can be combined; don't assume linearity
and normality

standardization: you can now blend r mode to q mode and vice versa, most
try to use a blend of both scenarios but, you can have issue with data
standardization, because mathematically large number always have
moreinfluence; for example: two species have different high number,
species 3 has higher number than specs3, rare taxa have almsot no weight
on outcome, maybe you might not want that, if you want to even up
contributions, we NEED to standardize for species abundance (this exm
might be easy). It might not be so easy for others. you can also
standardize your sides if they have different amount of abundancies. so
do you need to standardize coloumns(/max) and rows(/sum)? double
standardization is called:

normality: also important to consider normality. for this analysis squre
root transformation is ok, becase we never backtransform. the outcome is
unitless, so the log scale or normal scale doesn't matter.

Axis rotation or eigen analysis (p.7)

\url{http://www.flutterbys.com.au/stats/downloads/slides/pres.14.pdf}

3D display (x,y,z); by rotation of the axis, one axis (1) explains the
highest amount of the point, the next axis (2) explain the next amount
of variablity, same as the last axis (3), whidh doesn't have much noise
along this axis, hence we don't need this anymore, so could exclude that
one in order to explain most of the pattern. so i get a new variable on
axis1 from 3 to 1; this variable i can use in another analysis; thus, i
can test against rainfall or any other effect and then make a statement
about the community, if the y respond to that equally or not. the part
we missed axis (3), but it is the individual response, but not the major
pattern according to the community. PCA = principle analysis

Q mode analysis are by a row; dissimirity. how similar is site 2 to site
1 accroding to the species abundance pres 15, page 4: dissimilarity

calculate diddimilarity

euqulidian distance? disadvantage: when two things have nothing in
commont they are considered simliiar, no elefants, neg property, we
prefer what they have in common, advantage: it retains the units of the
measurements

bray curtis distance: based on 0 (completely the same) and 1 (completely
different), joint abstances are ignored,

Multidimensional scaling

dissimilarity is a triangle mirrow, that shows the similarity of the
sites based on the entire data we collected (1) create distance matrix
(genuine data) (2) choose dimensions: how many new axis do we want? and
it optimized for the best two or three axis (3) random configuration:
put the sites on a blank paper; correlate triangles to check on
correlation Stress: 1-R2 to check on the power; high r2 is 0.8, which is
a stress of \textless{}0.2 (4) measure: move your sites to improve your
site position, (5) (6) continual to iterate: thus you increate your
iterations until you lower your stress

non metric regression -\textgreater{} non metric regressional scale

When to stop your iterations?

\begin{enumerate}
\def\labelenumi{(\arabic{enumi})}
\tightlist
\item
  if your stress has a certain value, then stop, determine a threshold
  of the stress value
\item
  check (7)
\end{enumerate}

to combat what? (1) multiple random starts (2) you run a pco first to
get a rough idea where they should or could be and than do another
random start

\begin{enumerate}
\def\labelenumi{(\arabic{enumi})}
\setcounter{enumi}{6}
\tightlist
\item
  procrustes rotation: the axis mean nothing; staple plots of the sites
  to see how much i stretched the sites left or right; they line up
  pretty close; we sum up the ``residuals'', if it is leess than a
  certain number and then your ok; until you get a match, you stop
\end{enumerate}

function metaMDS does the following:

\begin{itemize}
\tightlist
\item
  if we haven't scaled or standardized the data, it will scale it
\item
  it generates dissimilarity, if you haven't done that
\item
  it runs then a PCoA to get the starting configuration
\item
  followed up by up to 20 random starts
\item
  because axis doesn't mean, it rotates the axis to get the
\end{itemize}

\begin{enumerate}
\def\labelenumi{(\arabic{enumi})}
\item
  R-Mode
\item
  Q-Mode
\end{enumerate}

Standardize to mean and variance (scale transformation - center and
scale):

Square root transformation is ok, because we don't want to backtransform
anyways, because the results of a multivariate analysis are
dimensionless and unitless.

\subsection{GLMM}\label{glmm}

Generalized linear mixed effects models (pres. 11.2a)

\subsection{GAM}\label{gam}

non-linear generalized additive models

\subsection{Import and Export}\label{import-and-export}

In order to \textbf{import} a dataset, you need to convert an Excel file
into a csv file.

To \textbf{export} from R into a publication,


\end{document}
